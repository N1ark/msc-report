\chapter{Implementation} \label{chap:implementation}

We now present the implementation of the state models and state model transformers presented. In \cref{sec:state-model-lib}, we present the library which exposes these state models, some relevant implementation details, and notable differences with the theory. In \cref{sec:impl-instantiations}, we then present the instantiations of Gillian for three different target languages using this library.

\section{State Model Library} \label{sec:state-model-lib}

All of the presented state models have been implemented in OCaml 5.2 \cite{ocaml} for Gillian \cite{gillian0,gillian1,gillian2}, a CSE engine that is parametric on the state model. We first present the general interface for state models, and how it connects to the Gillian engine. We then look more in depth into the implementation of two state models: \Ex{} for its canonical simplicity, and \PMap{} for its complexity. Finally, we list some additional ``utility'' state models that were omitted in the previous chapter but that are frequently used when instantiating state~models.

\subsection{State Model Interface}

State models are built as standalone \emph{modules}, while state model transformers are \emph{functors}, that accept one or more state model modules, and return a new state model module. Gillian already provides an interface for state models, \code{MonadicSMemory}, whose name originates from its extensive use of the \code{Delayed} monad: a \emph{symbolic execution monad} that allows branching and SMT queries in a sugared syntax that is easy to read and use, as seen in \autoref{fig:example-delayed}. The signature of the monad, in OCaml syntax, is \code{'a Delayed.t = Pc.t -> ('a * Pc.t) list}: given a PC, we return a list of branches, containing the new value and an updated PC. We make extensive use of \emph{let bindings}:\footnote{See \url{https://ocaml.org/manual/5.2/bindingops.html}} \code{let* v = d in e} is the \code{bind} operator, passing the value \code{v} contained in the delayed element \code{d} to the expression \code{e}, which must result in another delayed value. \code{let+} is the \code{map} operator, which modifies the value without branching. For the code shown, we branch when validating the index, and then branch again when calling \code{S.produce}; we then update the entry with \code{ss'}, without branching.

\begin{figure}
\begin{lstlisting}
let produce pred s args =
  match (pred, args) with
  | SubPred pred, idx :: args ->
    let* ss = validate_index s idx in
    let+ ss' = S.produce pred ss args in
    update_entry s idx ss'
\end{lstlisting}
\caption{Example of usage of the \code{Delayed} monad}
\label{fig:example-delayed}
\end{figure}

The interface provided by Gillian is not satisfactory: it is quite complex, requiring the implementation of 26 different functions whose purpose is at times unclear.\footnote{This is due to the fact Gillian has been in development for several years, and as it grew some unused functions were never cleaned up.} We thus create a simpler interface, \code{MyMonadicSMemory}, reducing the number of required functions and type definitions, and simplifying the definition of some functions (removing arguments when unneeded). Gillian also passes actions and core predicates as strings; while this is manageable for monolithic state models, this is unpractical. We instead require state models to define \code{action} and \code{pred} types, along with their \code{to\_string} and \code{from\_string} parsers, ensuring decoupling between parsing and execution. See \autoref{tab:monadicsmem-interface} for a side-by-side comparison of the interfaces. Because Gillian still requires the state model to satisfy \code{MonadicSMemory}, we provide a \code{Make} functor converting from our interface to Gillian's.

\begin{table}
\caption{Difference between Gillian's and our state model interface}
\centering
\begin{tabular}{|l|l|l|}
\hline
\textbf{Gillian interface} & \textbf{Our interface}  & \textbf{Outcome}  \\ \hline \hline
\code{apply\_fix}  &  & Removed  \\ \hline
\code{clean\_up}  &  & Removed  \\ \hline
\code{clear}  &  & Removed  \\ \hline
\code{copy}  &  & Removed  \\ \hline
\code{get\_failing\_constraint} &  & Removed  \\ \hline
\code{get\_init\_data}  &  & Removed  \\ \hline
\code{get\_print\_info}  &  & Removed  \\ \hline
\code{is\_overlapping\_asrt}  &  & Removed  \\ \hline
\code{mem\_constraints}  &  & Removed  \\ \hline
\code{pp\_by\_need}  &  & Removed  \\ \hline
\code{pp\_c\_fix}  &  & Removed  \\ \hline
\code{split\_further}  &  & Removed  \\ \hline
\code{sure\_is\_nonempty}  &  & Removed \\ \hline

\code{alocs}  & \code{alocs}  &  \\ \hline
\code{assertions}  & \code{assertions}  & Typed the core predicates  \\ \hline
\code{can\_fix}  & \code{can\_fix}  &  \\ \hline
\code{consume}  & \code{consume}  & Typed the core predicate  \\ \hline
\code{execute\_action}  & \code{execute\_action}  & Typed the action  \\ \hline
\code{get\_fixes}  & \code{get\_fixes}  & Simplified, use assertions \\ \hline
\code{get\_recovery\_tactic}  & \code{get\_recovery\_tactic} & Simplified  \\ \hline
\code{init}  & \code{empty}  & Renamed  \\ \hline
\code{lvars}  & \code{lvars}  &  \\ \hline
\code{pp}  & \code{pp}  &  \\ \hline
\code{pp\_err}  & \code{pp\_err}  &  \\ \hline
\code{produce}  & \code{produce}  & Typed core predicate  \\ \hline
\code{substitution\_in\_place}  & \code{substitution\_in\_place} &  \\ \hline

& \code{action\_from\_str}  & Added  \\ \hline
& \code{action\_to\_str}  & Added  \\ \hline
& \code{assertions\_others}  & Added  \\ \hline
& \code{compose}  & Added  \\ \hline
& \code{instantiate}  & Added  \\ \hline
& \code{is\_concrete}  & Added  \\ \hline
& \code{is\_empty}  & Added  \\ \hline
& \code{is\_exclusively\_owned}  & Added  \\ \hline
& \code{pred\_from\_str}  & Added  \\ \hline
& \code{pred\_to\_str}  & Added  \\ \hline
\end{tabular}
\label{tab:monadicsmem-interface}
\end{table}

Most of the removals are possible because some default function can be used for all state models, under the assumption they behave a particular way; for instance, we remove the \code{copy} function, because we assume all state models only work on immutable code. Most of the additions are due to the modular nature of our state models, or to enforce a stronger typing of actions and predicates.

A significant difference with the theory presented is that the \execac, \consume{} and \produce{} functions do not take as input states of type \code{t option} (to imitate $\Sst^?$), but instead \code{t}. While both options are possible, using the \code{option} monad proved to at times be somewhat unweildy (for the same reason it is unpractical to use $\fn{wrap}$ and $\fn{unwrap}$ in the rules) -- we instead rely on the \code{is\_empty} function when necessary. We also require a \code{compose} function, despite it not being justified in the theory, because of substitutions.

One of the most significant improvements of our interface is the re-work of fixes in the engine, to behave precisely like the definition of \fix{} in the theory. Gillian uses an intermediary \code{fix\_t} type, that is obtained from an error with \code{get\_fixes}, and must then be applied with \code{apply\_fix: t -> fix\_ t -> (t, err\_t) result Delayed.t}. Aside from being an unpractical system to use, generating a fix and then applying it separately, it \emph{introduces unsoundness into the bi-abductive engine}. Indeed, applying a fix modifies the state directly; there is no requirement for this fix to be describable in terms of assertions. This means that applying the same fix on the state and the anti-frame may give different results, that do not result in the appropriate change for the function's generated pre-condition.

For instance in \cite{sacha-phd}, an example implementation of \PMap{} is given, where executing an action on an empty cell creates a \Miss{}, called \code{MissingBinding}. A naive implementation of \PMap{} for Gillian could resolve this fix by simply adding a binding that points to the empty substate in the heap at the corresponding index. This however is not sound, as it \emph{cannot be expressed as an assertion}: it manipulates the representation of state, but no assertion exists that creates an empty binding, since \PMap{} doesn't define core predicates for bindings -- it only lifts the core predicates of the wrapped state model. This would therefore result in a fix that doesn't change the anti-frame or the specification of the function, which is invalid.

We fix this by enforcing \code{fix\_t} to be of type \code{MyAsrt.t}: strongly typed assertions. The implementation of \code{apply\_fix} then produces all the fixes into the state model. This change thus not only makes state models less prone to error, but also simplifies their code.

\subsection{Exclusive} \label{subsec:ex-impl}

We now present the implementation of the \Ex{} state model, step by step (skipping the uninteresting details such as parsing or auxiliary functions). Firstly, we must define the types for the state model: the state type, the errors (which include misses), the actions and core predicates. This is trivial: \begin{lstlisting}
type t = Expr.t option
type err_t = MissingState
type action = Load | Store
type pred = Ex
\end{lstlisting}
Here we define states as being an optional expression, where the \code{None} case stands for $\bot$. There is also one singular error, for misses. We then must define the function of the state model: \execac, \consume{} and \produce, following the rules outlined in \cref{rules:ex}: \begin{lstlisting}
let execute_action action s args =
  match (action, s, args) with
  | _, None, _ -> DR.error MissingState
  | Load, Some v, [] -> DR.ok (Some v, [ v ])
  | Store, Some _, [ v' ] -> DR.ok (Some v', [])
  | _ -> failwith "Invalid Ex execute_action"

let consume core_pred s ins =
  match (core_pred, s, ins) with
  | Ex, Some v, [] -> DR.ok (None, [ v ])
  | Ex, None, _ -> DR.error MissingState
  | _ -> failwith "Invalid Ex consume"

let produce core_pred s insouts =
  match (core_pred, s, insouts) with
  | Ex, None, [ v ] -> Delayed.return (Some v)
  | Ex, Some _, _ -> Delayed.vanish ()
  | _ -> failwith "Invalid Ex produce"
\end{lstlisting}
Again, this is straightforward; a simple pattern matching suffices. \code{DR.ok} stands for \mbox{\code{Delayed.return (Ok -)}}, returning the input in a \code{Result} monad, wrapped in a \code{Delayed} monad. The outcome is thus \Ok{} -- for \Err{}, \LFail{} and \Miss{} outcomes, we use \code{DR.error}; the state model then provides a \code{can\_fix} function that given an error returns whether it is fixable, signifying it's a \Miss. Experiments were done to create an \code{Outcome} monad, with variants \code{Ok | Miss | Err | LFail}, as a replacement for the \code{Result} monad with \code{Ok | Err}. While this allowed getting rid of the \code{can\_fix} function, it was unpractical to use, as it was purely ``aesthetic''; the engine still needed to parse everything back to a \code{Result}. Furthermomre, this made lifting errors quite tedious: instead of simply defining a variant \code{SubError of S.err\_t} for the error type and lifting the errors of the underlying state model through it, one had to handle all three error cases, creating a lot of boilerplate code for little to no added value.

Finally here we mention some of the additional functions that the state model must implement: \begin{lstlisting}
let empty () = None
let is_empty = Option.is_none
let is_exclusively_owned = Option.is_some
let instantiate = function
  | [] -> (Some (Expr.Lit Undefined), [])
  | [ v ] -> (Some v, [])
  | _ -> failwith "Invalid Ex instantiation"
let assertions = function
  | None -> []
  | Some v -> [ (Ex, [], [ v ]) ]
let can_fix MissingState = true
let get_fixes MissingState =
  [ [ MyAsrt.CorePred (Ex, [], [ LVar (Generators.fresh_svar ()) ]) ] ]
\end{lstlisting}
These functions are, in order: the constructor for the empty ($\bot$) state, a function to tell if the state is $\bot$, the \isexowned{} presented before, the \code{instantiate} function required by \PMap, a function to return all the core predicates associated with a state, the function to tell apart \Err{} and \LFail{} from \Miss{}, and the \fix{} function presented in \cref{chap:theory}. For \fix, we note that specifying the existential for the fix $\exists\sym x\ldotp \corepred{\exP}{}{\sym x}$ is not needed, as all unbound logical variables in Gillian are existentially quantified by default.

This is -- skipping the irrelevant functions -- all of the code for a simple state model in Gillian, using our interface \code{MyMonadicSMemory}. The entire code is 77 lines long, with the biggest function taking 9 lines. This shows that for simple state models, our interface allows for just as simple implementations.

\subsection{Partial Map}

We now present some details of the $\PMap(I,\mmdl)$ implementation. It is a functor, receiving a \code{PMapIndex} module and a \code{MyMonadicSMemory} module, corresponding to the $I$ and $\mmdl$ arguments of the constructor. Both \PMap{} and \DynPMap{} are implemented as one state model; what enables the ``dynamic'' behaviour is the value \code{mode: Static | Dynamic} that is contained in the input index. The only difference in behaviour this causes is enabling the \alloc{} action in static mode, and catching any \code{NotAllocated} errors and replacing them with an implicit allocation in dynamic mode. Aside from this, \code{PMapIndex} has an \code{is\_valid\_index}, which returns the ``actual'' index associated with an input index. This is where the abstract location optimisations plays a role: for the \code{LocationIndex} module, it is: \begin{lstlisting}
module LocationIndex : PMapIndex = struct
  let mode = Static

  let make_fresh () =
    let loc = ALoc.alloc () in
    Expr.loc_from_loc_name loc
    
  let is_valid_index = function
    | (Expr.Lit (Loc _) | Expr.ALoc _) as l -> Delayed.return (Some l)
    | e when not (Expr.is_concrete e) -> (
        let* loc = Delayed.resolve_loc e in
        match loc with
        | Some l -> Delayed.return (Some (Expr.loc_from_loc_name l))
        | None ->
            let loc' = make_fresh () in
            Delayed.return ~learned:[ (loc' #== e) ] (Some loc'))
    | _ -> Delayed.return None
end
\end{lstlisting}
We see here how \code{Delayed.resolve\_loc}, named \code{to\_aloc} in \cref{sec:theory-optim-pmap}, may not find a result, in which case we generate a new fresh abstract location to assign to the expression, by extending the PC: ``\code{\textasciitilde{}learned: [ (loc' \#== e) ]}'' appends the expression $\code{loc'} = \code{e}$ to it, with \code{\#==} signifying the symbolic equality.

This is only one part of the implementation for what was called \code{get} in the theory. The rest of it is two functions:\footnote{We omit some optimisations for clarity.} \begin{lstlisting}
let validate_index (h, d) idx =
  let* idx' = I.is_valid_index idx in
  match idx' with
  | None -> DR.error (InvalidIndexValue idx)
  | Some idx' -> (
      let* match_val = ExpMap.sym_find_opt idx' h in
      match (match_val, d) with
      | Some (idx'', v), _ -> DR.ok (idx'', v)
      | None, None -> DR.ok (idx', S.empty ())
      | None, Some d ->
          if%sat Formula.SetMem (idx', d) then DR.ok (idx', S.empty ())
          else DR.error (NotAllocated idx'))
\end{lstlisting}
This is the function in \PMap{} responsible for retrieving the substate associated to an index; it calls \code{ExpMap.sym\_find\_opt}, which symbolically tries to match an index against all entries of the map, and if that finds no result it either returns an empty substate $\bot$ with \code{S.empty~()}, or raises an error if a bound exists and $\sym i'\notin \sym d$. Here \code{if\%sat} is sugared syntax to allow branching according to a symbolic formula, using \code{Delayed}. If both the formula and its negation can be true then the value it returns evaluates to two branches, extended with the formula or its negation in the PC.
\begin{lstlisting}
module ExpMap = struct
  include Map.Make (Expr)
  
  let sym_find_opt k m =
    match find_opt k m with
    | Some v -> Delayed.return (Some (k, v))
    | None ->
        let rec find_match = function
          | [] -> Delayed.return None
          | (k', v) :: tl ->
              if%sat k' #== k then Delayed.return (Some (k', v))
              else find_match tl
        in
        find_match (bindings m)
end
\end{lstlisting}
This last function is the symbolic matching function, which checks if $\SAT(\sym i = \sym i')$ for every index $\sym i'$ in the map, and creates a branch for every possible match. It also does \emph{syntactic matching}, as seen at the start of the function: it first attempts to find the matching key directly, using \code{Map.find\_opt}, and only if that fails does it proceed to match with every key.

The \code{Delayed} monad and the associated syntactic sugar, initially described in \cite{sacha-phd}, allow us to write branching-heavy code with ease, abstracting the details of the PC away from us. We finally take a brief look at the implementation of \produce, to show how a state model transformer accesses the wrapped state model via the interface we defined: \begin{lstlisting}
let produce pred s args =
  match (pred, args) with
  | SubPred pred, idx :: args ->
      let*? idx, ss = validate_index s idx in
      let+ ss' = S.produce pred ss args in
      update_entry s idx idx ss'
  | DomainSet, [ d' ] -> (
      match s with
      | _, Some _ -> Delayed.vanish ()
      | h, None -> Delayed.return (h, Some d'))
  | _ -> failwith "Invalid PMap produce"
\end{lstlisting}
Here again, and even for a transformer as complex as $\PMap(I,\mmdl)$, the code is short and minimal. For the case of a core predicate from $\mmdl$ (\code{SubPred pred}), we get the substate associated with the index (\code{let*?} gets the value associated with a \code{Result} and vanishes if it's an error), we produce the predicate onto the substate, and then update the entry. For the domain set, we simply add it to the state, vanishing if a domain set is already present as it is exclusively owned.

The entire \PMap{} code is 343 lines of code -- a sizeable amount, but still rather short given its complexity.

\subsection{Helper State Models}

We now define some additional state model transformers that are often used when one wants to instantiate a state model with the aforementioned library. These transformers focus on applying small, customisable modifications to state models, allowing for an even greater re-use. In particular these utility state model transformers are useful when instantiating our state model with the goal to mimic a pre-existing state model.

$\statemodel{ActionAdd}(\mmdl, A)$ is a transformer that equips the given state model with additional actions defined by $A$, without needing to define any predicates, overriding functions, or re-implementing repetitive functions.

$\statemodel{Filter}(\mmdl, \actions, \Delta)$ is a transformer that allows filtering the actions and predicates exposed by a state model, for instance to limit the functionality of a state model if one of its provided actions shouldn't exist in the semantics of the language.

$\statemodel{Mapper}(\mmdl, F_\actions, F_\Delta)$ allows renaming predicates or actions.

$\statemodel{Injector}(\mmdl, I)$ adds hooks into \consume, \produce{} and \execac{}, allowing the developer to apply transformations to the inputs or outputs of these functions. For instance, this makes re-ordering the arguments of actions trivial, or allows treating an out-value as an in-value that makes the branch vanish if it doesn't correspond.

\section{Instantiations} \label{sec:impl-instantiations}

We now look at how this library of state models can be used to instantiate Gillian. In particular we focus on instantiating Gillian to target languages (TLs) that have already been instantiated for it: WISL, JavaScript and C.

To instantiate a language to Gillian, the developer must provide its state model, and a compiler from the TL to GIL. Because this project focuses on constructing state models, it is thus natural that we picked languages for which the compiler already exists.

We note that all forms of $\PMap(\Loc, \mmdl)$ implicitly make use of the abstract location optimisation described in \cref{sec:theory-optim-pmap}. This is because all of Gillian's instantiations also make use of it, and the goal of our instantiations is to have \emph{parity} with the reference implementation, rather than to have additional completeness. Indeed, while removing the unsoundness caused by ALocs is straightforward in our library, it causes a majority of tests to fail, making any comparison with Gillian impossible.

\subsection{WISL}

WISL (WhIle language for Separation Logic) is an imperative toy language used to teach students the principles of software verification. It uses a simple block-offset memory: \begin{align*}
 	\PMap(\Loc, \Freeable(\List(\Ex(\Val))))
\end{align*}

Unsurprisingly for a simple state model like this, its instantiation is also trivial when brought to code: \begin{lstlisting}
module ExclusiveNull = struct
  include Exclusive

  let instantiate = function
    | [] -> (Some (Expr.Lit Null), [])
    | [ v ] -> (Some v, [])
    | _ -> failwith "ExclusiveNull: instantiate: too many arguments"
end

module BaseMemory = PMap (LocationIndex) (Freeable (MList (ExclusiveNull)))

module WISLSubst : NameMap = struct
  let action_substitutions =
    [
      ("alloc", "alloc");
      ("dispose", "free");
      ("setcell", "store");
      ("getcell", "load");
    ]

  let pred_substitutions =
    [ ("cell", "points_to"); ("freed", "freed"); ("bound", "length") ]
end

module WISLMonadicSMemory = Mapper (WISLSubst) (BaseMemory)
\end{lstlisting}
This is \underline{all} the code needed for the instantiation of WISL to match exactly Gillian's original monolithic instantiation.

A consequence of our modular approach to state models is functor-heavy code, with many intermediary modules to allow for further constructions. This can be a bit unweidly if not organised properly, which explains why here we take some intermediary steps, instead of writing down one large and confusing functor application.\footnote{It is often not possible to write the whole state model as a single functor application anyways, due to constraints in OCaml abount naming modules -- see \url{https://github.com/ocaml/ocaml/issues/6917\#issuecomment-626586129}.}

We now describe, module by module, how the instantiation of WISL is done. We first override the \Ex{} state model, to modify its default value on instantiation, as WISL requires instantiation to \code{null} and the base implementation of \Ex{} initialises it to \code{undefined} instead. We then create the intermediary module \code{BaseMemory}, equivalent to $\PMap(\Loc,\Freeable(\List(\Ex)))$,\footnote{The \List{} state model transformer is called \code{MList} to avoid shadowing OCaml's \code{List} module.} which corresponds to the bulk of the state model. We finish by applying the substitutions needed to match the target instantiation, via the \code{Mapper} module, and voilà! Thanks to our library, what used to be about 800 lines of code is now 25 lines. We point out that our instantiation is also easier to verify the soundness of, as all modules are independent and don't require understanding how the other state model transformers work.

\subsection{JavaScript}

Gillian originated from JaVerT \cite{javert1,javert2}, which eventually became Gillian-JS, a JavaScript instantiation for Gillian, which targets ECMAScript 5.1 \cite{ecmascript}. Its state model is: \begin{align*}
	\PMap(\Loc, \DynPMap(\Str, \Ex(\Val)) \times \Ag(\Loc))
\end{align*}
Here $\DynPMap(\Str, \Ex(\Val))$ represents objects, and $\Ag(\Loc)$ is the location of the object's metadata, which itself is also an object. An object's metadata is immutable, and thus shareable, explaining the choice of an agreement.

Its implementation with our library is more involved than for WISL; indeed, Gillian-JS doesn't quite line up with the actions and predicates we defined for our state models. These differences mostly stem from different design, meaning most of the work required to fix them is juggling values to fit the right shape.

For instance in Gillian-JS the \domainset{} predicate has one in-value, the domain set, and no out. In our definitions however, the same predicate has no in, and one out, the domain set. To this end, we override \consume{} for \domainset{}, and vanish when the given in doesn't match the out.
\begin{lstlisting}
module MoveInToOut (S : States.MyMonadicSMemory.S) :
  States.MyMonadicSMemory.S with type t = S.t = struct
  include S

  let consume pred s ins =
    match (pred_to_str pred, ins) with
    | "domainset", [ out ] -> (
        let** s', outs = S.consume pred s [] in
        match outs with
        | [ out' ] ->
            if%sat out #== out' then Delayed_result.ok (s', [])
            else Delayed.vanish ()
        | _ -> Delayed_result.ok (s', outs))
    | _ -> consume pred s ins
end
\end{lstlisting}

Another difference is that for the \DynPMap, when accessing a field that didn't exist, Gillian-JS instantiates it to \code{Nono}, a special value to represent values that are not actually there. When using the \code{get\_domainset} action or consuming the \domainset{} predicate, all keys mapping to a \code{Nono} value must thus be removed:
\begin{lstlisting}
module PatchDomainsetObject (S : sig
  include MyMonadicSMemory
  val is_not_nono : t -> Expr.t -> bool
end) =
  Injector
    (struct
      include DummyInject (S)

      let post_execute_action a (s, args, rets) =
        match (a, rets) with
        | "get_domainset", [ Expr.EList dom ] ->
            let dom = List.filter (S.is_not_nono s) dom in
            Delayed.return (s, args, [ Expr.EList dom ])
        | _ -> Delayed.return (s, args, rets)

      let post_consume p (s, outs) =
        match (p, outs) with
        | "domainset", [ Expr.ESet dom ] ->
            let dom = List.filter (S.is_not_nono s) dom in
            Delayed.return (s, [ Expr.ESet dom ])
        | _ -> Delayed.return (s, outs)
    end)
    (S)
\end{lstlisting}
This is done via an injection, that receives a state model \emph{enhanced with the \code{is\_not\_nono} function}; we do this instead of hardcoding the lookup to ease the process of instantiating the JavaScript stack with different optimisations of \PMap{} that may not have the same state type. An example implementation is then:
\begin{lstlisting}
module ObjectBase = struct
  include PMap (StringIndex) (Exclusive)

  let is_not_nono (h, _) idx =
    match ExpMap.find_opt idx h with
    | Some (Some (Expr.Lit Nono)) -> false
    | _ -> true
end
\end{lstlisting}

The entire instantiation to JavaScript is 253 lines of code, including all versions of the state model with different optimisations enabled. Most of these lines are used on functor and module definitions, to build up the state one modification at a time. This type of code is hard to read and understand, but this complexity is not caused by the target state model as much as the need for matching the Gillian-JS interface; if the compiler from JavaScript to GIL was reworked to be more aligned with our state model definitions, the instantiation of JavaScript would likely be much simpler and succinct.

\subsection{C}

Gillian-C \cite{gillian0} targets a subset of C, via the CompCert-C verified compiler \cite{compcert}. Due to the intricacies of the C memory model, its construction uses custom components, not part of the shared library. Its state model is: \begin{align*}
	\PMap(\Loc, \Freeable(\BlockTree)) \times \CGEnv
\end{align*}

The first custom state model used is the \BlockTree{} component, which represents chunk of memorys as a binary tree, where each node has a range, a value, and zero or two children. Discussing the behaviour of this state model is out of the scope of this report, and is described in more detail in \cite{sacha-phd}.

The second custom state model is \CGEnv{}, the C global environment. Gillian-C handles function pointers and global variables by constructing an ``initial data'' when parsing the input code, and then passing this data to the state model on instantiation. While this behaviour was hard-coded in Gillian-C, we generalise it, considering simply that the state is a product of the memory and the global environment.

For the implementation of \BlockTree{} and \CGEnv, the code from Gillian-C was reused and adapted where necessary to our interface. A particular improvement that was done to it is that in Gillian-C the notion of \Freeable{} and of \BlockTree{} were coupled, meaning any action executed on the \BlockTree{} had to first check if the memory had been freed before proceeding. Decoupling this behaviour was straightforward, and allowed for simpler code in \BlockTree.

The bulk of the new code needed for the C instantiation lies in implementing a new action, that was previously hardcoded in the Gillian-C state model: \code{move}. This action allows copying some memory from a source into a target location, merging the \BlockTree{}s when appropriate. It is quite inelegant to implement, as it cannot rely on the abstraction provided by state models, since it needs to ``cross layers'' between the state models: it operates on two different indices of the \PMap, but applies a merge operation that only exists for \BlockTree. It seems that for such \emph{crossing actions}, our modular state model approach is not sufficient; one instead needs to implement these actions for a specific instantiation, making the proof work much more intricate.

Omitting this addition done via \statemodel{ActionAdd}, the rest of the state model is straightforward to instantiate: \begin{lstlisting}
module BaseMemory = PMap (LocationIndex) (Freeable (BlockTree))

module Wrap (S : C_PMapType) =
  Product
    (struct
      let id1 = "mem_"
      let id2 = "genv_"
    end)
    (ExtendMemory (S))
    (CGEnv)

module CMonadicSMemory = Wrap (BaseMemory)
\end{lstlisting}
Here \code{ExtendMemory} is the functor responsible for adding the \code{move} action to the state model. We make use of the \code{Wrap} functor to avoid code duplication when building the state model with different optimisations of \PMap{}. The module defined on lines 5-8 specifies the suffixes used to distinguish actions and core predicates from the left and right side of the product.

The instantiation of C with our library is a good example of mixing modular, general use state models with specialised state models; one can still make use of this library while implementing a perhaps more performant or more complex state model for some of the construction.








