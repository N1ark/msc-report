\chapter{Introduction}

\section{Motivation}

As software becomes increasingly central to critical infrastructure, the need to formally prove its correctness increases. Symbolic execution tools have been developed to allow for the verification of programs automatically, though with poor scalability \cite{surveysymex}. 

 A framework that enables scalable program verification is \emph{separation logic} \cite{seplogic1,seplogic2}, by allowing one to reason about fragments of the state. Properties are shown to hold for a part of memory, and from this any disjoint memory can be composed with it while guaranteeing the properties still hold.

\emph{Compositional symbolic execution} (CSE) uses separation logic to allow for the verification of functions independently from the rest of the code, improving drastically scalability \cite{pathexplo} and allowing for its adoption in Continuous Integration pipelines \cite{pulse}.

A variety of CSE tools exist to verify different code written in different languages, such as C \cite{verifast,infer,pulse}, Java \cite{jstar,verifast,infer,pulse}, JavaScript \cite{javert1} or intermediate languages \cite{corestar,viper,gillian0}. For this project we focus on Gillian \cite{gillian1,gillian2}; it has the unique property of being \emph{parametric on the state model}, allowing it to be used to verify any language, if its state model is provided by the developer. A state model is a partial commutative monoid (PCM), endowed with actions, core predicates, and functions operating on it to modify it. It supports over-approximate and under-approximate reasoning, using separation logic and incorrectness separation logic, respectively. Gillian has succesfuly been instantiated to JavaScript, C and Rust \cite{gillian0,gillianrust}, enabling code verification, true bug finding, and whole program symbolic testing. These instantiations are monolithic, implemented as one tightly coupled piece of code, and make use of specific optimisations to improve performance, which are never justified in the theory.

From Gillian stemmed research on an abstract CSE engine \cite{cse1,cse2}; it defines at the theory level what properties must be satisfied by a state model for the engine to be sound, and provides a mechanised proof of the soundness of the engine.

While separation logic reasons about a specific memory model, the most typical example of which being a linear heap, \emph{abstract} separation logic is a branch of separation logic that allows one to reason about properties of the memory independently of its actual representation \cite{higherorderseplogic,abstractseplogic}. An outstanding logic in this field is Iris \cite{iris}, a certified logic that is implemented in Rocq (formerly Coq) \cite{coq} and which proved itself to be capable of enabling reasoning for a wide range of applications. In particular, Iris introduces the \emph{resource algebra} (RA), an algebraic structure that allows \emph{building up more complex RAs from simpler elements} while carrying the soundness properties of the underlying elements. These logics have however been limited to manual pen and paper proofs, or mechanisations with Rocq, both requiring extensive technical and theoretical knowledge.

Gillian is particularly well suited for this concept of constructing complex state from simpler elements: because it is parametric on the state model, it can already technically support any construction provided to it. In \cite{sacha-phd}, the idea of constructing state models from simpler elements was thus ported to Gillian, however using PCMs, the traditional choice of structure for abstract separation logic \cite{abstractseplogic,sepalgebra,iris1,higherorderseplogic}. Only definitions were given however, with no usable implementation alongside it. Furthermore, the use of PCMs limited what could be done with these constructions, and introduced some soundness issues. More specifically, PCMs enforce a unit, which can become problematic when considering frame preserving actions and some particular constructions.

We are interested in bridging the gap between Iris and Gillian with regards to state model constructions, by replacing the PCMs used with a structure more akin to Iris's resource algebras. This should simplify the proof of soundness for state model transformers, while bringing the theory of an abstract CSE engine closer to the theory developed in Iris. In particular we shall interest ourselves on what changes need to be applied to this abstract CSE engine to function with RAs.

We shall then revisit the state model and state model transformer definitions that were first defined in \cite{sacha-phd} and that were ported from Iris \cite{iris}, bringing them back to their original RA representation while adapting the additional functions that make them state models to use RAs. Furthermore, we are interested in seeing if we can bring the optimisations that already exist in monolithic Gillian instantiations to our state model transformers \emph{within the theory}, proving their soundness, which has never been attempted before in this context.

While having sound constructions is essential, Gillian is before all a tool that should be used. We must thus see if the state model definitions we establish in the theory and prove the soundness of can be transferred to Gillian with sufficiently good performance, while permitting the construction of state models that can be used to verify real world languages. We are also interested in seeing how we may reproduce the state models for C and JavaScript using state model transformers, allowing for sound constructions without the burden of proving this soundness.

\section{Contributions}

The main contribution of this project is the adoption of resource algebras for the construction of state models that are compatible with CSE engines.

We define at the theory level seven different state models and state model transformers, which are all proven to be sound. These can then be combined seamlessly to build up complex state models that are sound by construction.

We also define in the theory three optimisations of the partial map transformer that originally only existed in the implementation of Gillian, proving the soundness of two of them.

We then implement these state models along with additional ones in OCaml, creating a state model library compatible with the CSE engine Gillian. Using this library, we instantiate Gillian to allow for the verification of code written in WISL, JavaScript and C. These instantiations have full parity with the pre-existing instantiations of Gillian, providing the same verification results with a fraction of the code needed for the instantiation.

Finally, we measure in great detail the performance of the instantiations and optimisations described, comparing them to the performance of the monolithic instantiations; we show that our constructions are, for C and JavaScript, more performant, while requiring significantly less code and proof effort.

\section{Outline}

In \autoref{chap:background} we introduce the necessary theoretical background needed for the project; we go over separation logic and its variations, with incorrectness separation logic and abstract separation logic. In particular we retrace the history of abstract separation logic and its evolution, leading to the creation of Iris. We then describe Gillian, a compositional symbolic execution engine parametric on the memory model. We list existing tools that enable symbolic execution and their differences. Finally, we briefly cover additional requirements relevant to this research area.

In \autoref{chap:theory} we motivate the transition from partial commutative monoids to resource algebras. We then go through the changes brought to our abstract CSE engine to accommodate the change, and define three different state models and four different state model transformers. We also cover optimisations at the theoretical level for the partial map transformer.

In \autoref{chap:implementation} we describe the implementation of these state models as a library, and how our theoretical constructs interface with the pre-existing definitions of Gillian. We also cover how a developer may use this library to instantiate Gillian with the desired state model.

In \autoref{chap:evaluation} we evaluate the work done so far, discussing first the usability of the theory and of the library. We then compare the performance of our instantiations with the reference Gillian instantiations. We also briefly observe the performance improvements offered by our partial map optimisations.

In \autoref{chap:conclusion} we conclude this report, retracing the work done, and outlining the key achievements, the relevant limitations, and give different directions to build upon what is defined here, for future work.
